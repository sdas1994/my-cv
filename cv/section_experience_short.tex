% Awesome Source CV LaTeX Template
%
% This template has been downloaded from:
% https://github.com/darwiin/awesome-neue-latex-cv
%
% Author:
% Christophe Roger
%
% Template license:
% CC BY-SA 4.0 (https://creativecommons.org/licenses/by-sa/4.0/)

%Section: Work Experience at the top
\sectionTitle{Work Experience}{\faSuitcase}
%\renewcommand{\labelitemi}{$\bullet$}
\begin{experiences}
  \experience
    {Current}       {Project: AskAVA | Lead Engineer, Product Owner}{Arista Networks}{Feb 22 to present}
    {February 2024} {
                      \begin{itemize}
                        \item Designed an LLM based chat interface for network security analysts to interact with all resources
                        \item Employed reasoning capabilities of LLMs to decide correct function calling approaches, given a prompt
                        \item Implemented RAG approaches over documentation to supercharge reasoning ability of LLMs
                        \item Experimented with different vector stores for document chunk embeddings, evaluating on performance
                        \item Utilized LlamaIndex as a framework to abstract ingestion, storing, querying and evaluation
                      \end{itemize}
                    }
                    {OpenAI,Gemini,LLM,Vertex AI,Milvus,FAISS,LlamaIndex,Docker,VS Code}
  \emptySeparator
  \experience
    {December 2023} {Project: IoC++ | Lead Engineer, Product Owner}{Arista Networks}{Feb 22 to present}
    {January 2024}  {
                      \begin{itemize}
                        \item Designed an interactive platform to enable threat researchers gain an holistic view on IoCs
                        \item Employed multithreading and asynchronous approaches in Python to populate data in an efficient manner
                        \item Created a Postgres database to store pulled data, and a schema for efficient updation and retrieval
                        \item Built FastAPI endpoints for accessing said data, and a Streamlit frontend to run custom queries
                      \end{itemize}
                    }
                    {FastAPI,Python,OpenAPI,Postgres,Multithreading,AsyncIO,cron,Streamlit}
  \emptySeparator
  \experience
    {March 2022}    {Project: DeviceAVA | Lead Engineer, Product Owner}{Arista Networks}{Feb 22 to present}
    {November 2023} {
                      \begin{itemize}
                        \item Designed a system of gRPC microservices in Python, responsible for device fingerprinting 
                        \item Asynchronous integration with third party APIs, using multithreading, to maximize performance
                        \item Designed a novel schema to populate local signatures, based on regexes and Jinja templates
                        \item Dockerized the build process, used Terraform to programatically control deployment of container on AWS
                        \item Highly scalable, handles a million concurrent requests with consistent 99.99th percentile latency
                        \item Built metrics pipeline and dashboard on Prometheus and Grafana, and ran analytics on activity data
                        \item Led a team of two to develop data pipelines on GPDB, orchestrated by Apache Airflow using Scala
                        \item Supervised building of a database of devices with auxilliary information. Created an endpoint to match string query to closest matching database entry using string similarity and semantic embedding
                      \end{itemize}
                    }
                    {gRPC,Python,Hyperscan,Jinja,Docker,AWS Fargate,ALB,Terraform,Airflow,GPDB,Prometheus,Grafana}
  \emptySeparator
  \experience
    {December 2019} {Project: Catalog Recommendation | Data Scientist II}{Meesho}{Dec 19 to Feb 22}
    {February 2022} {
                      \begin{itemize}
                        \item Modeled a collaborative filtering based recommender engine using ALS method for Matrix Factorization
                        \item Implemented strategies to manage catalog views in context aware manner to boost serendipity
                        \item Performed offline benchmarking on metrics like nDCG, followed by A/B testing on significant users
                        \item Implemented DropoutNet architecture over the MF model to tackle cold-start problem, using Tensorflow
                        \item Developed Airflow maintained Apache Spark jobs for data ingestion, and Redis cache, for backend service
                        \item Personalized item feed led to improved KPI for the affected real estate, views/session grew by upto ~17\%
                      \end{itemize}
                    }
                    {Recommendation System,DropoutNet,Collaborative Filtering,Apache Spark,Airflow,Tensorflow,Redis}
  \emptySeparator
  \experience
  {October 2018}  {Project: DCO | MTS Software Developer I/II}{Adobe Inc.}{July 17 to Dec 19}
  {December 2019} {
                    \begin{itemize}
                      \item Modeled the problem of Dynamic Creative Optimization as a Contextual Multi-armed Bandit Problem to progressively model user affinity towards different ad creatives
                      \item Devised a novel approach by incorporating Field-aware Factorization Machines into the ε-greedy framework, to handle advertising data with 99\% sparsity and 99.9\% class imbalance.
                      \item Developed data pipelines for effectively data cleaning, merging and feature engineering in Apache Spark
                    \end{itemize}
                  }
                  {Factorization Machines,Reinforcement Learning,Dynamic Creative Optimization,Apache Spark,Multi-armed Bandit}
\end{experiences}
